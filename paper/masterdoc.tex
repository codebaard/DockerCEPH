\documentclass[titlepage, a4paper, 11pt]{scrartcl}

%too much whitespace otherwise
\usepackage[left=2cm,right=2cm,top=2cm,bottom=2cm]{geometry}

% Grafik Pakete
\usepackage{graphicx,hyperref,amssymb}

% Ordner für Grafiken
\graphicspath{ {./images/} }
\usepackage{float}

\usepackage[utf8]{inputenc}
\usepackage{amsmath}
\usepackage{amsfonts}
\usepackage{amssymb}
\usepackage{graphicx}

\usepackage{caption}
\usepackage{subcaption}

% Header and Footer
\usepackage{fancyhdr}

%bibtex
\usepackage{cite}

%code snippets
\usepackage{listings}
\usepackage{color}

\definecolor{dkgreen}{rgb}{0,0.6,0}
\definecolor{gray}{rgb}{0.5,0.5,0.5}
\definecolor{mauve}{rgb}{0.58,0,0.82}

\lstset{frame=tb,
    language=HTML,
    aboveskip=3mm,
    belowskip=3mm,
    showstringspaces=false,
    columns=flexible,
    basicstyle={\small\ttfamily},
    numbers=none,
    numberstyle=\tiny\color{gray},
    keywordstyle=\color{blue},
    commentstyle=\color{dkgreen},
    stringstyle=\color{mauve},
    breaklines=true,
    breakatwhitespace=true,
    tabsize=3
}

\pagestyle{fancy}
\fancyhf{}
\rhead{Julius Neudecker, 2025850}
\lhead{CEPH Cluster in containers}

\title{Running a CEPH-Cluster on a containerized infrastructure}
\subtitle{Use case: distributed mySQL-database}
\author{Julius Neudecker \\ Bachelor of Science \\ \href{mailto:julius.neudecker@haw-hamburg.de}{julius.neudecker@haw-hamburg.de}}
\date{January 2020}


\begin{document}

    \maketitle

    \tableofcontents

    \begin{abstract}
        Setting up and operate a storage cluster with high availability is a complex task. Modern paradigmas like containerization
        and orchestration are a way of abstracting away some complexity. However, running a cluster in a stateless and ephemeral
        containerized environment poses some problems. In the following paper these problems are identified and scrutinized.
        The use case will be a mySQL database, which will be stored on a CEPH cluster comprised of docker based daemons.
    \end{abstract}

    \section{Introduction}
        %Managing a highly available storage cluster is not a trivial task. Apart from provisioning and monitoring the hardware,
        %setting up multiple systems concurrently is a daunting task. Nowadays with provisioning tools like Salt, Chef or Puppet, 
        %this is easier than ever. However

        %Docker, stateless, ephemeral vs database, acid, ...
        \subsection{CEPH Based storage cluster}

        \subsection{Provisioning}

        \subsection{Containerization}

        \subsection{Databases}

        \subsection{Scope of the problem}

    \section{Setting up CEPH on Docker}

        \subsection{System Architecture}

        \subsection{Monitor Nodes}

        \subsection{Object Storage Devices - OSD}
        
    \section{Setting up the database}

        \subsection{Structure of mySQL}

        \subsection{ACID}

        \subsection{Problems with clusters}

    \section{Performance}

        \subsection{Integrity}

        \subsection{Penalty}

        \subsection{Administration}

    \section{Conclusion}        

    \bibliography{literature}        
    \bibliographystyle{IEEEtran}

\end{document}